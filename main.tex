\documentclass{article}
\usepackage[utf8]{inputenc}

\title{SpecialTopics: CompChem}
\author{Coray Colina, Alberto Perez, Adrian Roitberg }
\date{October 2019}

\begin{document}

\maketitle

\section{Who is our target audience?}
We are targeting graduate computational students who want a deeper understanding about Molecular Dynamics/Monte Carlo techniques and issues/possibilities for advanced sampling techniques.

This course should be useful to students from Chemistry, Physics and Material Science.

As a special topics class, it will not count for requirements for Core or out of division classes, as such we will need to encourage our students to attend.

\section{Desired outcomes}
We want to take our students from the practical application of MD to mastering what are the underlying assumptions and approximations.

As a result students will be able to read the methods in a computational paper and know  if the simulations were carried out correctly and hence weather or not all the results/discussions are meaningful.

Students will become more aware of the possibilities and limitations in computational methods by looking at errors, convergence and efficiency of sampling methods. This will tie in between (advanced) sampling methods and ensemble analysis tools and protocols.

\section{Topics to cover}
\begin{itemize}
    \item Force fields
    \item Integrators
    \item Shake
    \item Timescales
    \item Barriers / multiple minima
    \item Enhanced sampling
    \item Statistics

    \item Toy system: HP lattice models, Ala dipeptide or similar:
	    \begin{itemize}
	        \item Exhaustive (enumeration in HP) sampling 
	        \item MD (different timescales)
	        \item REMD
	        \item Other advanced sampling techniques
	    \end{itemize}
    \item How do you optimize REMD performance?
    \item $\Delta$G calculations in biomolecular systems: path independent vs path dependent
	\item Umbrella sampling
	\item Accelerated MD
    \item Markov State Models: 
        \begin{itemize} 
            \item As ensemble processing
	        \item As adaptive sampling tools
        \end{itemize}
\end{itemize}

\end{document}
